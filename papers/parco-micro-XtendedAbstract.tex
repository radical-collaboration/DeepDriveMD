\documentclass[conference,final]{IEEEtran}

\usepackage[colorlinks]{hyperref}
\usepackage[normalem]{ulem}
\usepackage[utf8]{inputenc}
\usepackage{amsmath}
\usepackage{amssymb}
\usepackage{booktabs}
\usepackage{color}
\usepackage{fancyvrb}
\usepackage{float}
\usepackage{graphicx}
\usepackage{ifpdf}
\usepackage{keyval}
\usepackage{listings}
\usepackage{moresize}
\usepackage{multirow}
\usepackage{rotating}
\usepackage{setspace}
\usepackage{subfigure}
\usepackage{url}
\usepackage{wrapfig}
\usepackage{xcolor}
\usepackage{xspace}

\definecolor{listinggray}{gray}{0.95}
\definecolor{darkgray}{gray}{0.7}
\definecolor{commentgreen}{rgb}{0, 0.4, 0}
\definecolor{darkblue}{rgb}{0, 0, 0.4}
\definecolor{middleblue}{rgb}{0, 0, 0.7}
\definecolor{darkred}{rgb}{0.4, 0, 0}
\definecolor{brown}{rgb}{0.5, 0.5, 0}

\makeatletter
\def\cyanuwave{\bgroup \markoverwith{\lower3.5\p@\hbox{\sixly \textcolor{cyan}{\char58}}}\ULon}
\def\reduwave{\bgroup \markoverwith{\lower3.5\p@\hbox{\sixly \textcolor{red}{\char58}}}\ULon}
\def\blueuwave{\bgroup \markoverwith{\lower3.5\p@\hbox{\sixly \textcolor{blue}{\char58}}}\ULon}
\font\sixly=lasy6
\makeatother

\AtBeginDocument{
  \hypersetup{
    citecolor=blue,
    linkcolor=blue,   
    urlcolor=blue}}
\newif\ifdraft
\drafttrue
\ifdraft
\definecolor{ocolor}{rgb}{1,0,0.4}
\newcommand{\onote}[1]{ {\textcolor{ocolor} { (***Ole: #1) }}}
\newcommand{\terminology}[1]{ {\textcolor{red} {(Terminology used: \textbf{#1}) }}}
\newcommand{\owave}[1]{ {\cyanuwave{#1}}}
\newcommand{\jwave}[1]{ {\reduwave{#1}}}
\newcommand{\alwave}[1]{ {\blueuwave{#1}}}
\newcommand{\jhanote}[1]{ {\textcolor{red} { ***shantenu: #1 }}}
\newcommand{\alnote}[1]{ {\textcolor{blue} { ***andreL: #1 }}}
\newcommand{\amnote}[1]{ {\textcolor{blue} { ***andreM: #1 }}}
\newcommand{\georgenote}[1]{ {\textcolor{brown} { ***sharath: #1 }}}
\newcommand{\revThreeNote}[1]{ {\textcolor{purple}{}}} 
\newcommand{\revOneNote}[1]{ {\textcolor{purple}{}}} 
\newcommand{\revTwoNote}[1]{ {\textcolor{purple}{}}} 
\definecolor{orange}{rgb}{1,.5,0}
\newcommand{\aznote}[1]{ {\textcolor{orange} { ***ashley: #1 }}}
\definecolor{dandelion}{cmyk}{0,0.29,0.84,0}
\newcommand{\mtnote}[1]{ {\textcolor{dandelion} { ***matteo: #1 }}}
\newcommand{\note}[1]{ {\textcolor{magenta} { ***Note: #1 }}}
\else
\newcommand{\onote}[1]{}
\newcommand{\terminology}[1]{}
\newcommand{\owave}[1]{#1}
\newcommand{\jwave}[1]{#1}
\newcommand{\alnote}[1]{}
\newcommand{\amnote}[1]{}
\newcommand{\athotanote}[1]{}
\newcommand{\georgenote}[1]{}
\newcommand{\pmnote}[1]{}
\newcommand{\jhanote}[1]{}
\newcommand{\msnote}[1]{}
\newcommand{\mrnote}[1]{}
\newcommand{\aznote}[1]{}
\newcommand{\mtnote}[1]{}
\newcommand{\note}[1]{}
\newcommand{\revOneNote}[1]{}
\newcommand{\revTwoNote}[1]{}
\newcommand{\revThreeNote}[1]{} 
\fi

\newcommand{\cloud}{cloud\xspace}
\newcommand{\clouds}{clouds\xspace}
\newcommand{\pilot}{Pilot\xspace}
\newcommand{\pilots}{Pilots\xspace}
\newcommand{\pilotjob}{Pilot-Job\xspace}
\newcommand{\pilotjobs}{Pilot-Jobs\xspace}
\newcommand{\pilotcompute}{Pilot-Compute\xspace}
\newcommand{\pilotcomputedescription}{Pilot-Compute Description\xspace}
\newcommand{\pilotdescription}{Pilot-Description\xspace}
\newcommand{\pilotcomputes}{Pilot-Computes\xspace}
\newcommand{\pilotdata}{Pilot-Data\xspace}
\newcommand{\pilotdatainmem}{Pilot-Data Memory\xspace}
\newcommand{\pilotdatadescription}{Pilot-Data Description\xspace}
\newcommand{\pilotdataservice}{Pilot-Data Service\xspace}
\newcommand{\pilotcomputeservice}{Pilot-Compute Service\xspace}
\newcommand{\computedataservice}{Compute-Data Service\xspace}
\newcommand{\computedatamanager}{Compute-Data Manager\xspace}
\newcommand{\computeunitdescription}{Compute-Unit Description\xspace}
\newcommand{\dataunitdescription}{Data-Unit Description\xspace}
\newcommand{\pilotmapreduce}{PilotMapReduce\xspace}
\newcommand{\mrmg}{MR-Manager\xspace}
\newcommand{\pstar}{P*\xspace}
\newcommand{\pd}{PD\xspace}
\newcommand{\pc}{PC\xspace}
\newcommand{\pcs}{PCs\xspace}
\newcommand{\pj}{PJ\xspace}
\newcommand{\pjs}{PJs\xspace}
\newcommand{\pds}{Pilot Data Service\xspace}
\newcommand{\computeunit}{Compute-Unit\xspace}
\newcommand{\computeunits}{Compute-Units\xspace}
\newcommand{\dataunit}{Data-Unit\xspace}
\newcommand{\dataunits}{Data-Units\xspace}
\newcommand{\du}{DU\xspace}
\newcommand{\dus}{DUs\xspace}
\newcommand{\dud}{DUD\xspace}
\newcommand{\cu}{CU\xspace}
\newcommand{\cus}{CUs\xspace}
\newcommand{\cud}{CUD\xspace}
\newcommand{\su}{SU\xspace}
\newcommand{\sus}{SUs\xspace}
\newcommand{\schedulableunit}{Schedulable Unit\xspace}
\newcommand{\schedulableunits}{Schedulable Units\xspace}
\newcommand{\cc}{c\&c\xspace}
\newcommand{\CC}{C\&C\xspace}
\newcommand{\up}{\vspace*{-1em}}
\newcommand{\upp}{\vspace*{-0.5em}}
\newcommand{\numrep}{8 }
\newcommand{\samplenum}{4 }
\newcommand{\tmax}{$T_{max}$ }
\newcommand{\tc}{$T_{C}$ }
\newcommand{\tcnsp}{$T_{C}$}
\newcommand{\bj}{BigJob\xspace}
\newcommand{\irods}{iRODS\xspace}

\newcommand{\I}[1]{\textit{#1}\xspace}
\newcommand{\B}[1]{\textbf{#1}\xspace}
\newcommand{\T}[1]{\texttt{#1}\xspace}

\lstdefinestyle{myListing}{
  frame=single,
  backgroundcolor=\color{listinggray},
  language=C,
  basicstyle=\ttfamily \footnotesize,
  breakautoindent=true,
  breaklines=true
  tabsize=2,
  captionpos=b,
  aboveskip=0em,
  belowskip=-2em,
}

\lstdefinestyle{myPythonListing}{
  frame=single,
  backgroundcolor=\color{listinggray},
  language=Python,
  basicstyle=\ttfamily \scriptsize,
  breakautoindent=true,
  breaklines=true
  tabsize=2,
  captionpos=b,
}

\ifpdf
\DeclareGraphicsExtensions{.pdf, .jpg, .tif}
\else
\DeclareGraphicsExtensions{.ps,  .eps, .jpg}
\fi

\tolerance=1000
\hyphenpenalty=10

\lstnewenvironment{code}[1][]%
{
\noindent
%\minipage{0.98 \linewidth}
\minipage{1.0 \linewidth}
\vspace{0.5\baselineskip}
\lstset{
    language=Python,
%    numbers=left,
%    numbersep=4pt,
    frame=single,
    captionpos=b,
    stringstyle=\ttfamily,
    basicstyle=\scriptsize\ttfamily,
    showstringspaces=false,#1}
}
{\endminipage}



\begin{document}

%\title{Integrating Convolutional Autoencoders with Molecular Dynamics
%Simulations on HPC Platforms}
\title{Deep Generative Model driven Protein Folding Simulations on HPC Platforms}
\author{Vivek Balasubramanian$^{1}$, Debsindhu Bhowmik$^{2}$, Jumana Dakka$^{1}$, Shantenu Jha$^{1}$, \\ Hyungro Lee$^{1}$, Heng Ma$^{2}$, Arvind Ramanathan$^{4}$, Matteo Turilli$^{1}$, Michael T. Young$^{2}$ \\
   {\footnotesize{\emph{$^{1}$RADICAL, ECE, Rutgers University, Piscataway,NJ 08854, USA}}}\\
   \footnotesize{\emph{$^{2}$}}Computational Sciences and Engineering, Oak Ridge National Laboratory, Oak Ridge, TN 37830\\
   \footnotesize{\emph{$^{3}$}}Data Science and Learning, Argonne National Laboratory, Lemont, IL 60439\\
   %\footnotesize{\emph{$^{4}$}\upp\upp\upp}
   }


\date{}
\maketitle

% ---------------------------------------------------------------------------
\section{Scientific  and Computational Motivation}

Multiscale molecular simulations are widely used to model complex biological phenomena, such as protein folding, protein-ligand (e.g., small molecule, ligand/ drug, protein) interactions, and self-assembly. However, much of these phenomena occur at timescales that are fundamentally challenging for molecular simulations to access, even with advances in both hardware and software technologies. Hence, there is a need to develop scalable, adaptive simulation strategies that can enable sampling of timescales relevant to these biological phenomena. 

Many adaptive sampling techniques have been proposed and these techniques share some similar characteristics, including (a) the need for (efficient) automated approaches to identify a small number of relevant conformational coordinates (either through clustering and/or dimensionality reduction techniques), and (b) the identification of the ‘next’ set of simulations to run such that more trajectories are successful in attaining a specific end goal (e.g., protein that is well folded, protein bound to its target ligand, etc.). While there are numerous approaches to cluster simulations (such as Markov State Models and variational approach for molecular processes) to characterize transition pathways from ensembles of bio-molecular simulations, we recently developed a deep learning based approach that uses convolutions and a variational autoencoder (CVAE) to cluster simulations in an unsupervised manner~\cite{bhowmik2018deep}. We have shown that our CVAE can discover intermediate states from protein folding pathways; further, the CVAE-learned latent dimensions cluster conformations into biophysically relevant features (such as number of native contacts, or root mean squared deviation to native state). 

We posit that the latent features learned by the CVAE can be used to drive adaptive sampling within molecular dynamics (MD) simulations, where the next set of simulations to run are decided based on a measure of ‘novelty’ of the simulation/ trajectory frame observed. In this paper, we implement our deep learning driven adaptive sampling framework within the RADICAL-Ensemble toolkit~\cite{balasubramanian2018harnessing} to specify and execute a workflow with multiple instances of MD simulations and CVAEs. Our contributions can be summarized as follows: 
\begin{itemize}
\item We demonstrate that deep learning based approaches can be used to drive adaptive MD simulations at scale. We demonstrate our approach in folding small proteins and show that it is possible to fold them in a small number of iterations of the adaptive sampling than using traditional approaches. 
\item We highlight how the workflow characterization is quite unique as the training of deep learning algorithms can take almost as much time as running simulations, necessitating novel developments within RADICAL-Ensemble toolkit to deal with resource allocation, scheduling and management. 
\end{itemize}


% ---------------------------------------------------------------------------
\section{Methods} 

Two key components in the workflow are MD simulation and CVAE. To fully
utilize the computational power of Summit with NVIDIA Tesla V100
accelerators, we explicitly choose software compiled for GPUs to carry out
these two set of tasks.


\subsection{Molecular Dynamics simulation}

The MD simulations are performed on GPUs with OpenMM
7.3.0~\cite{eastman2017openmm}. The Fs-peptide system is described with 
Amberff99sb-ildn force field in implicit Onufriev-Bashford-Case GBSA solvent 
model. The non-bonded 
interactions are cut off at 1.0 nm and no periodic boundary condition is
applied. All the bonds to hydrogen are fixed to their equilibrium value to
enable 2 fs time step. Langevin integrator is used to maintain the system
temperature at 300 K with friction coefficient at 1 ps$^{-1}$. Other than
trajectories, a new reporter is added to the simulation that calculates the 
contact matrix of protein $C_{\alpha}$s using 
MDAnalysis~\cite{michaud2011mdanalysis,gowers2016mdanalysis} 
module and output it into hdf5 format. 

\subsection{Convolutional Variational Autoencoder}

Autoencoder is a deep neural network architecture that can represent high
dimensional data in a low dimensional latent space while retaining the key
information. With its unique hourglass shaped architecture, an autoencoder
compresses input data into a latent space with reduced dimension and
reconstructs it to the original data. Since output of the network is the
reconstruction of input features, it can handle the unlabeled data sets and
capture essential information in the latent space. Additionally, the
variational layer constraints the data points to a normal distribution in
latent space, in which way the latent embeddings will be evenly distributed
and it links to any points in latent space to patterns in the original
dataset. Convolutional layers are added before the feedforward layers,
applying a filter to the input contact maps, which can improve the robustness
of the network in recognizing the local patterns that represents local
interactions between C-alpha from neighboring residues regardless of their
positions.~\cite{bhowmik2018deep}  Each CVAE neural network is constructed with Keras/TensorFlow~\cite{chollet2015keras,abadi2016tensorflow}
packages and trained on GPU for 100 epochs.

\begin{figure*}
	\centering
	\includegraphics[width=.8\textwidth]{MicroScope_Workflow_Diagram}
	\caption{The workflow diagram. Each frame in MD simulations is equivalent to 50 ps. }
	\label{fig:microscopeworkflowdiagram}
\end{figure*}

% ---------------------------------------------------------------------------
\section{Software and Platforms}

EnTK exposes an application programming interfaces that enables users to
specify workflows in terms of pipelines, stages and tasks. Each pipeline is
composed of a sequences of stages, and each stage is a set of tasks.
Sequences and set encode the execution priority among tasks: stage \#2 must
execute after stage \#1 but all tasks of a stage can execute concurrently.
Each task encapsulates a program, not a method or a function.

EnTK uses RADICAL-Pilot (RP) as its runtime system~\cite{merzky2018using}. RP
is a pilot systems, i.e., it enables the decoupling between the acquisition
of HPC resources and the scheduling of tasks on those
resources~\cite{turilli2018comprehensive}. RP acquires resources by
submitting a job to the batch system of the target resource and then uses a
private scheduler to schedule tasks on those resources. In this way, tasks do
not have to wait on the resource batch queue to be executed, enabling
high-throughput on high-performance computing resources within the boundaries
of the fair usage policies of the target machine.

Fig.~\ref{fig:microscopeworkflowdiagram} shows a diagrammatic representation
of our workflow. We use the OpenMM simulation engine to execute our MD
simulations. Each OpenMM executable runs on a single GPU and simulates an
independent physical system. Our CVAE implementation gets as input a
aggregated set of simulations' trajectories and uses it to train its model.
Multiple trainings are concurrently repeated until, upon successful inference
of the hyperspace, CVAE output is used to drive a new set of simulations.

Both OpenMM and CVAE run as self-contained, independent executables that, in
turn, can use multithreading and/or multiprocessing. In this way, each task
can use multiple CPUs or GPUs as needed. Currently, our workflow uses 18
concurrent OpenMM instances and requires 18 GPUs. The CVAE implementation
uses 1 GPUs and the whole hyerspace optimization takes $2^5$ CVAE trainings
for a total of 32 GPUs. Our workflow is implemented in EnTK with two
pipelines, one for MD simulations and the other for CVAE trainings and
hyperspace optimization. The two pipelines execute concurrently, each
requiring at every point in time a minimum of 18 and 32 GPUs respectively.

We executed our workflow on Summit, the new leadership-class machine managed
by OLCF at ORNL\@. Currently, Summit is the largest HPC machine in the world,
offering 44 CPU cores, each with 4 hardware threads, and 6 GPUs per node.


% ---------------------------------------------------------------------------
\section{Results and Future Work}

We show preliminary results that indicates the ability to integrate the
simulation component of the workflow into RADICAL-Cybertools, and the
relatively low overheads of our tools.  As a baseline, we profile the
performance of RADICAL-Cybertools using a single OpenMM instance on ORNL
Summit. Specifically, we benchmark the FS-Peptide system and run the
simulations for 100 ns on a single Tesla V100 GPU. 

Fig.~\ref{fig:single_openmm} shows the task execution time as reported in the
OpenMM logs and the overheads of EnTK and RP, measured using
RADICAL-Analytics, our in-house profiling tool. OpenMM tasks executed for
5837s $\sigma$20.3, while EnTK overhead was 1.2s $\sigma$0.05, and
RADICAL-Pilot overhead 5837s $\sigma$5.8. From this baseline we can
conclude that RADICAL-Cybertools overheads are negligible and that we are
ready to move to executing the workflow at scale.

\begin{figure*}
    \centering
    \includegraphics[width=.8\textwidth]{single_openmm}
    \caption{}
    \label{fig:single_openmm}
\end{figure*}

As part of the final paper, we plan to demonstrate two aspects of our AI-driven workflow: 
\begin{itemize}
\item \emph{Scaling characteristics}: We will show how larger proteins can be folded by  training across multiple GPUs (using a data parallel implementation of the CVAE). Additionally, we will demonstrate how hyperparameter optimization using our HyperSpace~\cite{Young_2018} can improve the robustness of the CVAE while achieving scalability on the Summit supercomputer. We will also demonstrate how increasing the data load (through the number of concurrently instantiated trajectories, as well as building larger CVAE models distributed across nodes) can affect the performance of our workflow. 
\item \emph{Providing near real-time inference for CVAE}: A critical component of our workflow is to instantiate newer simulations of novel conformations (inferred from our CVAE model) as simulations are still running. This will involve enabling building near real time inference of the best trained / optimized CVAE such that novel conformations can be identified across all simulations currently running. 
\end{itemize}
We believe that the performance characteristics of our workflow within the RADICAL-EnTK will provide quantitative underpinnings for challenges involved in implementing AI-driven multiscale molecular simulations at scale on emerging HPC platforms. 

\bibliographystyle{unsrt}
\bibliography{parco-micro}

\end{document}